\documentclass{article}
\usepackage[spanish]{babel}
\usepackage[onehalfspacing]{setspace}
\usepackage[utf8]{inputenc}
\usepackage{amsmath}
\usepackage{amssymb}
\usepackage{verbatim}
\usepackage{graphicx}
\usepackage{listings}
\usepackage{fullpage}
\usepackage{color}
\usepackage{fancyvrb}
\usepackage{hyperref}
\hypersetup{%
	pdfborder = {0 0 0}
}

\definecolor{mygreen}{rgb}{0,0.6,0}
\definecolor{mygray}{rgb}{0.5,0.5,0.5}
\definecolor{mymauve}{rgb}{0.58,0,0.82}

\lstset{ %
	backgroundcolor=\color{white},   % choose the background color; you must add \usepackage{color} or \usepackage{xcolor}
	basicstyle=\footnotesize,        % the size of the fonts that are used for the code
	breakatwhitespace=false,         % sets if automatic breaks should only happen at whitespace
	breaklines=true,                 % sets automatic line breaking
	captionpos=b,                    % sets the caption-position to bottom
	commentstyle=\color{mygreen},    % comment style
	frame=single,                    % adds a frame around the code
	keepspaces=true,                 % keeps spaces in text, useful for keeping indentation of code (possibly needs columns=flexible)
	numbers=left,                    % where to put the line-numbers; possible values are (none, left, right)
	numbersep=5pt,                   % how far the line-numbers are from the code
	numberstyle=\tiny\color{mygray}, % the style that is used for the line-numbers
	rulecolor=\color{black},         % if not set, the frame-color may be changed on line-breaks within not-black text (e.g. comments (green here))
	showspaces=false,                % show spaces everywhere adding particular underscores; it overrides 'showstringspaces'
	showstringspaces=false,          % underline spaces within strings only
	showtabs=false,                  % show tabs within strings adding particular underscores
	stepnumber=1,                    % the step between two line-numbers. If it's 1, each line will be numbered
	stringstyle=\color{mymauve},     % string literal style
	tabsize=4,
	title=\lstname                   % show the filename of files included with \lstinputlisting; also try caption instead of title
}


\author{José Luis Cánovas Sánchez\\joseluis.canovas2@um.es\\48636907A}
\title{ARQUITECTURAS DE REDES AVANZADAS\\PRÁCTICA 2\\Balanceo de Carga con SDN\\ ENERO 2015}
\date{}
\begin{document}
\maketitle

\begin{abstract}

\end{abstract}

\tableofcontents

\section{Introducción}

\section{Topología mininet}


\section{Controlador POX con l2\_learning}


\begin{BVerbatim}
$sudo mn --custom topo.py --topo mytopo --controller remote --test pingall
	*** Creating network
	*** Adding controller
	*** Adding hosts:
	cli_1 cli_2 cli_3 cli_4 cli_5 cli_6 srv_1 srv_2 srv_3 srv_4
	*** Adding switches:
	s1 s2
	*** Adding links:
	(cli_1, s1) (cli_2, s1) (cli_3, s1) (cli_4, s1) (cli_5, s1) (cli_6, s1) (s1, s2) (srv_1, s2) (srv_2, s2) (srv_3, s2) (srv_4, s2)
	*** Configuring hosts
	cli_1 cli_2 cli_3 cli_4 cli_5 cli_6 srv_1 srv_2 srv_3 srv_4
	*** Starting controller
	c0
	*** Starting 2 switches
	s1 s2 ...
	*** Waiting for switches to connect
	s1 s2
	*** Ping: testing ping reachability
	cli_1 -> cli_2 cli_3 cli_4 cli_5 cli_6 srv_1 srv_2 srv_3 srv_4
	cli_2 -> cli_1 cli_3 cli_4 cli_5 cli_6 srv_1 srv_2 srv_3 srv_4
	cli_3 -> cli_1 cli_2 cli_4 cli_5 cli_6 srv_1 srv_2 srv_3 srv_4
	cli_4 -> cli_1 cli_2 cli_3 cli_5 cli_6 srv_1 srv_2 srv_3 srv_4
	cli_5 -> cli_1 cli_2 cli_3 cli_4 cli_6 srv_1 srv_2 srv_3 srv_4
	cli_6 -> cli_1 cli_2 cli_3 cli_4 cli_5 srv_1 srv_2 srv_3 srv_4
	srv_1 -> cli_1 cli_2 cli_3 cli_4 cli_5 cli_6 srv_2 srv_3 srv_4
	srv_2 -> cli_1 cli_2 cli_3 cli_4 cli_5 cli_6 srv_1 srv_3 srv_4
	srv_3 -> cli_1 cli_2 cli_3 cli_4 cli_5 cli_6 srv_1 srv_2 srv_4
	srv_4 -> cli_1 cli_2 cli_3 cli_4 cli_5 cli_6 srv_1 srv_2 srv_3
	*** Results: 0\% dropped (90/90 received)
	*** Stopping 1 controllers
	c0
	*** Stopping 11 links
	...........
	*** Stopping 2 switches
	s1 s2
	*** Stopping 10 hosts
	cli_1 cli_2 cli_3 cli_4 cli_5 cli_6 srv_1 srv_2 srv_3 srv_4
	*** Done
	completed in 6.529 seconds
\end{BVerbatim}

\section{Construcción del balanceador de carga}

\subsection{Módulo de balanceo en POX}

\subsection{Modificación topología}

\subsection{Ejecución y prueba de PING}

\subsection{Tráfico y flujos}


\begin{thebibliography}{99}


\end{thebibliography}


\end{document}
